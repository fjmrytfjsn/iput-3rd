\documentclass{ltjsarticle}

\title{
    活性化関数・学習率・バッチサイズの違いによる
    CNN分類モデルの学習比較実験レポート
}
\author{藤村勇仁}
\date{\today}

\usepackage[top=25truemm, bottom=25truemm, left=25truemm, right=25truemm]{geometry}
\usepackage{amsmath, amssymb}
\usepackage{siunitx}
\usepackage{pdfpages}
\usepackage{here}
\usepackage{diagbox}
\usepackage{slashbox}
\usepackage{tikz}
\bibliographystyle{MyStyle}

%---------------------------------------------------------------------

%番号付き箇条書き
\usepackage{enumerate}
\renewcommand{\theenumi}{\arabic{enumi}}
\renewcommand{\labelenumi}{\theenumi.}
\renewcommand{\theenumii}{\theenumi.\arabic{enumii}}
\renewcommand{\labelenumii}{\theenumii.}

%---------------------------------------------------------------------

%ハイパーリンク
\usepackage[unicode,hidelinks,pdfusetitle]{hyperref}
\hypersetup{
    setpagesize=false,
    colorlinks=false,
    pdftitle={},
    pdfsubject={},
    pdfauthor={},
    pdfkeywords={},
    pdfencoding=auto
}

%---------------------------------------------------------------------

\setcounter{secnumdepth}{6}
\setlength\textfloatsep{2truemm}

%---------------------------------------------------------------------

\fontsize{11ptj}{16ptj}\selectfont
\setlength{\baselineskip}{16pt}
\setlength{\columnsep}{5mm}

%---------------------------------------------------------------------

\newcommand{\TableOfContents}{
    \pagenumbering{roman}
    \tableofcontents
    \newpage
    \pagenumbering{arabic}
}

%---------------------------------------------------------------------

%表
\usepackage{tabularx}
\newcolumntype{C}{>{\centering\arraybackslash}X}
\newcolumntype{L}{>{\raggedright\arraybackslash}X}
\newcolumntype{R}{>{\raggedleft\arraybackslash}X}

%---------------------------------------------------------------------

%Listings
\usepackage{listings}
\lstset{
    basicstyle={\ttfamily},
    identifierstyle={\small},
    commentstyle={\small\itshape},
    keywordstyle={\small\bfseries},
    ndkeywordstyle={\small},
    stringstyle={\small\ttfamily},
    frame={tb},
    tabsize=4,
    breaklines=true,
    columns=[l]{fullflexible},
    numbers=left,
    xrightmargin=0\zw,
    xleftmargin=3\zw,
    numberstyle={\scriptsize},
    stepnumber=1,
    numbersep=1\zw,
    lineskip=-0.5\zh
}
\newcommand{\code}[3]{
    \lstinputlisting[label={code:#1}, caption={#2}]{#3}
}
\renewcommand{\lstlistingname}{コード}

%---------------------------------------------------------------------
% \makeatletter
% \def\Hline{
% \noalign{\ifnum0=`}\fi\hrule \@height 2pt \futurelet
%     \reserved@a\@xhline
% }
% \makeatother
\newcommand{\Hline}{\noalign{\hrule height 2pt}}

%キャプション(番号無し)
\newcommand{\subcaption}[1]{
    #1 \\
    \vskip\baselineskip
}

%表の空欄
\newcommand{\blank}{\textbf{---}}

%数式(番号付き)
\newcommand{\eq}[1]{
    \begin{eqnarray}
        #1
    \end{eqnarray}
}
%数式(番号無し)
\newcommand{\EQ}[1]{
    \begin{eqnarray*}
        #1
    \end{eqnarray*}
}

%一階微分
\newcommand{\diff}[2]{\frac{d #1}{d #2}}
%二階微分
\newcommand{\DIFF}[2]{\frac{d^2 #1}{d {#2}^2}}

%図(1つ)
\newcommand{\fig}[4][0.9]{
    \begin{figure}[H]
        \centering
        \includegraphics[width=#1\linewidth]{#4}
        \caption{#3}\label{fig:#2}
    \end{figure}
}
%図(2つ)
\newcommand{\ffig}[6]{
    \begin{figure}[H]
        \centering
        \begin{minipage}{0.45\linewidth}
            \centering
            \includegraphics[width=\linewidth]{#4}
            \caption{#3}\label{fig:#3}
        \end{minipage}
        \begin{minipage}{0.45\linewidth}
            \centering
            \includegraphics[width=\linewidth]{#6}
            \caption{#5}\label{fig:#2}
        \end{minipage}
    \end{figure}
}
%図(3つ)
\newcommand{\fffig}[9]{
    \begin{figure}[H]
        \centering
        \begin{minipage}{0.3\linewidth}
            \centering
            \includegraphics[width=\linewidth]{#4}
            \caption{#3}\label{fig:#3}
        \end{minipage}
        \begin{minipage}{0.3\linewidth}
            \centering
            \includegraphics[width=\linewidth]{#6}
            \caption{#5}\label{fig:#6}
        \end{minipage}
        \begin{minipage}{0.3\linewidth}
            \centering
            \includegraphics[width=\linewidth]{#8}
            \caption{#7}\label{fig:#2}
        \end{minipage}
    \end{figure}
}

\begin{document}
\begin{titlepage}
    \centering
    \vspace*{\stretch{4}}
    \textbf{\LARGE 令和7年度 深層学習}
    \vspace{\stretch{1}} \\
    \textbf{\LARGE
        MNISTの学習における \\
        全結合層と畳み込み層の違いが \\
        学習時間と精度に与える影響の調査
    }
    \vspace{\stretch{2}} \\
    \large 令和7年5月14日
    \vspace{10pt} \\
    \large 大阪国際工科専門職大学 \\
    \large 工科学部 \\
    \large 情報工学科 \\
    \large AI開発コース
    \vspace{10pt} \\
    \large OK240100 藤村勇仁
    \vspace{\stretch{6}}
\end{titlepage}


\TableOfContents

\section{はじめに}
本実験では、畳み込みニューラルネットワーク(CNN)において、活性化関数(ReLU, sigmoid)、学習率、バッチサイズの違いが学習速度や検証精度に与える影響を比較・検討した。最適なハイパーパラメータを探索することは、モデルの性能向上に不可欠であるため、その基礎的知見を得ることを目的とした。

\section{実験方法}
モデルにはPyTorchを用いたSimpleCNNを採用した。活性化関数としてはReLUおよびsigmoidを中間層に適用し、出力層には活性化関数を用いずクロスエントロピー誤差で損失を算出した。学習率は0.01、0.005、0.0001、0.00005を、バッチサイズは32および64を選択した。

データセットにはKaggleで公開されている「Dogs vs. Cats」データセット\footnote{\url{https://www.kaggle.com/c/dogs-vs-cats}}を利用し、犬と猫の画像を2クラスに分類する課題設定とした。Kaggle上の公式データセットをダウンロードし、PyTorchのDataLoaderを用いて学習・検証データとして読み込んだ。エポック数は5または10とした。

実験に使用したコードは、\ref{sec:code}部に示す。

\section{実験結果}
図\ref{fig:result}に、総学習時間と精度の関係を示す。
\fig[]{result}{実験結果の散布図}{../compare.png}
学習率0.01および0.005では、ReLU・sigmoidのどちらを用いても学習が進まず、エポック5時点の検証精度は0.5前後に留まった。例えば学習率0.01, バッチサイズ32, ReLUの場合、エポック5までの総学習時間は約279.96秒、検証精度は0.5014であった。sigmoidの場合も同様に、総学習時間283.59秒、検証精度0.4986と変わらなかった。学習率0.005でも、ReLUは279.50秒、sigmoidは300.09秒で、いずれも検証精度は0.5014であった。

一方、学習率を0.0001や0.00005まで下げると、ReLUの場合に限り学習が進み、検証精度が向上した。学習率0.0001, ReLU, バッチサイズ32の条件では、5エポックでの総学習時間は296.99秒、検証精度は0.7880となった。sigmoidではこの条件でも精度が0.4976に留まり、依然として学習が進まない傾向が見られた。バッチサイズを64に増やした場合、エポックごとの学習時間は短縮しつつも、検証精度は大きく変化しなかった。

\section{考察}
今回の実験から、学習率が高すぎる場合には損失が減少せず、検証精度も向上しないことが明らかとなった。特にsigmoidを活性化関数に用いた場合、勾配消失などの影響もあり、学習率を下げても十分な精度向上が得られなかった。ReLUでは学習率を適切に下げることで損失が減少し、検証精度も0.78程度まで上昇した。バッチサイズの増加による学習時間短縮効果は確認できたが、今回は精度への顕著な影響は見られなかった。

\section{まとめ}
活性化関数としてReLUを用い、学習率を0.0001程度まで下げることで、CNNモデルは安定して高い精度を発揮できることが確認できた。一方、sigmoidでは学習が進まず、検証精度も上がらない点から、深層学習においてはReLUの有効性が再確認された。今後は層構造や正則化手法、学習率スケジューラの導入なども視野にさらなる性能向上を目指す。

\section{参考データ}
表\ref{tab:result}に各条件でのエポック5時点の総学習時間と検証精度をまとめる。

\begin{table}[h]
\centering
\begin{tabular}{|c|c|c|c|c|}
\hline
活性化関数 & 学習率 & バッチサイズ & 総学習時間(秒) & 検証精度 \\
\hline
ReLU & 0.01   & 32 & 279.96 & 0.5014 \\
sigmoid & 0.01   & 32 & 283.59 & 0.4986 \\
ReLU & 0.005  & 32 & 279.50 & 0.5014 \\
sigmoid & 0.005  & 32 & 300.09 & 0.5014 \\
ReLU & 0.0001 & 32 & 296.99 & 0.7880 \\
sigmoid & 0.0001 & 32 & 331.55 & 0.4976 \\
\hline
\end{tabular}
\caption{各条件でのエポック5時点の総学習時間と検証精度}
\label{tab:result}
\end{table}

\section{使用コード}\label{sec:code}

\code{dataset}{データセット整形}{../separate.py}

\code{model}{CNNモデル定義(PyTorch)}{../model.py}

\code{trainloop}{学習}{../train_with_time.py}

\code{experiment}{実験の実行}{../experiment.py}

\code{summarygraph}{総学習時間と精度の関係グラフ描画}{../compare_exp.py}

\end{document}

